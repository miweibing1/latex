% !Mode:: "TeX:UTF-8"
\section{课题背景及研究的目的和意义}
\subsection{课题背景}
(正文  宋体小4号字,多倍行距值1.25,段前0行,段后0行。字数3000字以上。
具体的撰写要符合哈尔滨工业大学本科生毕业论文撰写规范的书写规定。)
\cite{hithesis2017}\inlinecite{cnproceed}

在现代医学影像技术飞速发展的背景下,医学图像已成为疾病诊断、治疗规划和预后评估
的核心依据。然而,单一模态医学图像往往受限于成像原理的固有缺陷,难以全面反
映人体组织的解剖结构与生理功能信息。例如,计算机断层扫描(CT)图像虽能清晰
呈现骨骼等高密度组织的结构细节,但对软组织的分辨能力有限;磁共振成像(MRI)可
提供高对比度的软组织信息,却在钙化灶检测中表现不足;正电子发射断层扫
描(PET)能反映组织代谢活性,却缺乏精确的解剖定位。因此,多模态医学图像融合技
术应运而生,其通过整合不同模态图像的互补信息,为临床诊断提供更全面、准确的视
觉依据,已成为医学图像处理领域的研究热点。

\subsection{研究的目的和意义}
\section{国内外在该方向的研究现状及分析}
\subsection{国外现状及分析}
\subsection{国内现状及分析}
\section{研究内容及拟解决的关键问题}
\subsection{研究内容}
\subsection{拟解决的关键问题}
\section{拟采取的研究方法和技术路线、进度安排、预期达到的目标}
\subsection{拟采取的研究方法和技术路线}
\subsection{进度安排}
\subsection{预期达到的目标}
\section{课题已具备和所需的条件}
\section{研究过程中可能遇到的困难和问题,解决的措施}
\section{参考文献}
\bibliographystyle{hithesis}
\bibliography{reference}

% Local Variables:
% TeX-master: "../report"
% TeX-engine: xetex
% End: